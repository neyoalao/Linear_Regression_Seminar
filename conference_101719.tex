\documentclass[conference]{IEEEtran}
%\IEEEoverridecommandlockouts
% The preceding line is only needed to identify funding in the first footnote. If that is unneeded, please comment it out.
\usepackage{cite}
\usepackage{amsmath,amssymb,amsfonts}
\usepackage{algorithmic}
\usepackage{graphicx}
\usepackage{textcomp}
\usepackage{xcolor}
\def\BibTeX{{\rm B\kern-.05em{\sc i\kern-.025em b}\kern-.08em
    T\kern-.1667em\lower.7ex\hbox{E}\kern-.125emX}}
\begin{document}

\title{Linear Regression: Deep Learning}

\author{\IEEEauthorblockN{Olaniyi Bayonle Alao}
\IEEEauthorblockA{\textit{Summer Term, 2021} \\
\textit{Bachelor of Electronic Engineering} \\
\textit{Hochschule Hamm-Lippstadt}\\
Lippstadt, Germany \\
olaniyi-bayonle.alao@stud.hshl.de}
}

\maketitle

\begin{abstract}
this paper talks about linear regression in the context of deep learning. Linear regrression is a statictical term that uses a dependent and independent variable to make predictions. in these paper, the use case of linear and multiple regression is used to predict data in machine learning. The data were gotten from kaggle.com. the data is a data of the titanic survivers and the model was trained using sci-kit or tensor flow.

data cleasing was done using pandas a python library to prepare the data for model training.

regression analysis as a whole is a stastical method used to understand the relation between dependent and independent variables.


with the rise in the amount of data we have access to and increase in the performance of computers, Machine Learning a subset of Artificial inteligence has seen a significant growth. machine learning refers to  is the ability of applications to get better at doing things without necessarily a change in the code base them is something that has really been helpful. 

even though machine learning isn't a new concept, it is fast gaining recognization and changing lives thanks to the increase in the processing power of computers over the years to be able to process big data at a level that has never been experienced.


\end{abstract}

\begin{IEEEkeywords}
machine learning, linear regression, deep learning, sci-kit, tensor flow
\end{IEEEkeywords}

\section{Introduction}
This document is a model and instructions for \LaTeX.
Please observe the conference page limits. 

\begin{itemize}
\item talk about the different types of regression but focus on multiple linear regression.
\item Linear regression theorectical background
\item method used in the paper
\item application case analysis ==/$>$ algorithm implementation ==> results
\item conclusion
%\end{itemize}

\section{Theoretical Background}

\subsection{Machine Learning}
talk briefly about 
\item machine learning....types of it and what it is even about
\item deep learning.....what is it about?

\subsection{Regression}
talk about and also dive deep into the mathematics involved....
\item how regression came about ..... Francis something
\item types of regression ...... linear and multiple
\item what is it used for + how is it used?
\item use in the context of this paper i.e machine learning


\subsection{Model Evaluation}
\item R\textsuperscript{2}
\item talk about k-fold cross-validation....used to determine how well the model worked

\subsection{Libraries and Tools used}
talk about 
\item python and the library been used.....matplotlib, pandas, numpy\
\item machine and deep learning frameworks in python
\item describe the relevant parts of tensorflow/sci-kit.... what are they applied to and mapping the code to the regression formular?
\item for now thinking of using sci-kit as there is much tutorials based on it.

\section{Pratical Example}
\item framework that will be used and maybe why it is being used
\item preparing data for modelling.......removing useless data and reason for doing so!
\item coding to trrain the model and test it
\end{itemize}
\section{Result}
talk about the results of the findings

\section{Conclusion}
finish by highlighting the strenth of linear regression as well as the shorcomings.

\end{document}
